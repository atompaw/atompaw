\documentclass[11pt]{article}
\usepackage{url}
\special{papersize=8.5in,11in} \pagestyle{plain}
\setlength{\oddsidemargin}{0in} \setlength{\evensidemargin}{0in}
\setlength{\textwidth}{6.5in} \setlength{\parindent}{0in}
\setlength{\parskip}{0.2in} \setlength{\topmargin}{-0.2in}
\setlength{\headsep}{0in} \setlength{\headheight}{0in}
\setlength{\textheight}{9.5in} \setlength{\jot}{0.2in}
\usepackage{amsmath}




\begin{document}

\begin{center}
{{{\bf{Notes for 4.2.0.0 version of {\textsc{atompaw}} code.}}}} \\
{N. A. W. Holzwarth, Wake Forest University, Winston-Salem, NC. \\
 \today}
\end{center}

In the \textsc{atompaw} Release 4.2.0.0, several changes from
earlier versions have been introduced.    

\begin{itemize}
\item Marc Torrent constructed a new module {\verb+input_dataset_mod.F90+}
which now takes care of all of the input and stores the relevant
parameters in a datastructure.
\item Natalie Holzwarth implemented the capability for graphatom
to self-consistently solve the Dirac equation.   At the moment, this
capability is not yet integrated into the atompaw and PAW
formalism.
\item Natalie Holzwarth, Marc Torrent, and Michel C\^{o}t\'{e} 
completed a reasonable version of the code to converge the
generalized Kohn-Sham equations for meta-GGA functionals including
the kinetic energy density.     This came with a ``splinesolver"
algorithm for the self-consistent radial equations that replaces
the inward and outward integrations to find bound state 
radial wavefunctions.    The splinesolver is used by default
for meta-GGA functionals, but can be requested for other 
functionals.   Some of the pseudoization schemes have been
adopted for developing self-consistent PAW datasets appropriate
for meta-GGA functionals within the generalized Kohn-Sham approach.
Details of this implementation are published -- PRB (2022).
\end{itemize}



With these changes come new options for the input files for
\textsc{graphatom} and \textsc{atompaw}.


Some examples input data for \textsc{graphatom} and their
corresponding summary output [Atom].GA and input data for
\textsc{atompaw} and the corresponding summary output [Atom] are as follows.

\fbox{Example 1: \textsc{graphatom} input for Si -- LDA, non-relativistic}
\begin{verbatim}
Si 14
LDA-PW loggrid    2001
3 3 0 0 0 0
3 1 2
0 0 0
END

####Si.GA####
 Completed calculations for Si
 Perdew-Wang LDA -- PRB 45, 13244 (1992)
  Radial integration grid is logarithmic 
r0 =   4.3309254E-04 h =   6.0632956E-03   n =      2001 rmax =   8.0000000E+01
 Non-relativistic calculation
   AEatom converged in          28  iterations
     for nz =  14
     delta  =    4.0395229093591495E-017

 Orbital energies
 n  l     occupancy            energy
 1  0      2.0000000E+00 -1.3036860E+02
 2  0      2.0000000E+00 -1.0149628E+01
 3  0      2.0000000E+00 -7.9623458E-01
 2  1      6.0000000E+00 -7.0293986E+00
 3  1      2.0000000E+00 -3.0661972E-01

  Total energy
     Total                    :    -576.38746219636232     
######################################
\end{verbatim}


\fbox{Example 2: \textsc{graphatom} input for Si -- LDA, non-relativistic,
using splinesolver}
\begin{verbatim}
Si 14
LDA-PW splineinterp  splr00.1d0  splns600 loggrid    2001
3 3 0 0 0 0
3 1 2
0 0 0
END

###Si.GA###
 Completed calculations for Si
 Perdew-Wang LDA -- PRB 45, 13244 (1992)
  Radial integration grid is logarithmic 
r0 =   4.3309254E-04 h =   6.0632956E-03   n =      2001 rmax =   8.0000000E+01
 Splinesolver used for bound states
Splinesolver method used for bound states
Splinesolver grid parameters r0 and ns:
   0.10000      600
 Non-relativistic calculation
   AEatom converged in          23  iterations
     for nz =  14
     delta  =    2.2774037112883901E-016

 Orbital energies
 n  l     occupancy            energy
 1  0      2.0000000E+00 -1.3037233E+02
 2  0      2.0000000E+00 -1.0149191E+01
 3  0      2.0000000E+00 -7.9616715E-01
 2  1      6.0000000E+00 -7.0296405E+00
 3  1      2.0000000E+00 -3.0664292E-01

  Total energy
     Total                    :    -576.38791341463855     
#########################################
\end{verbatim}


\fbox{Example 3: \textsc{graphatom} input for Si -- r2SCAN01, non-relativistic,
using generalized Kohn-Sham}
\begin{verbatim}
Si 14
WTAU XC_MGGA_X_R2SCAN01+XC_MGGA_C_R2SCAN01 splr00.1d0  splns600 loggrid 2001
3 3 0 0 0 0
3 1 2
0 0 0
END

###Si.GA###
 Completed calculations for Si
 Using Libxc -- XC_MGGA_X_R2SCAN01+XC_MGGA_C_R2SCAN01
  Radial integration grid is logarithmic 
r0 =   4.3309254E-04 h =   6.0632956E-03   n =      2001 rmax =   8.0000000E+01
 Splinesolver used for bound states
Splinesolver method used for bound states
Splinesolver grid parameters r0 and ns:
   0.10000      600
 Non-relativistic calculation
   AEatom converged in          19  iterations
     for nz =  14
     delta  =    3.9767450008378184E-017

 Orbital energies
 n  l     occupancy            energy
 1  0      2.0000000E+00 -1.3175185E+02
 2  0      2.0000000E+00 -1.0501811E+01
 3  0      2.0000000E+00 -8.1811500E-01
 2  1      6.0000000E+00 -7.1862608E+00
 3  1      2.0000000E+00 -3.0245126E-01

  Total energy
     Total                    :    -578.62529162965529     
########################################
\end{verbatim}



\fbox{Example 4: \textsc{graphatom} input for Si -- Dirac equation}
\begin{verbatim}
Si 14
LDA-PW diracrelativistic loggrid    2001
3 3 0 0 0 0
3 1 -2  0.d0
3 1  1  2.d0
0 0 0  0
END

###Si.GA
 Completed calculations for Si
 Perdew-Wang LDA -- PRB 45, 13244 (1992)
  Radial integration grid is logarithmic 
r0 =   4.3309254E-04 h =   6.0632956E-03   n =      2001 rmax =   8.0000000E+01
 Dirac-relativistic calculation
   AEatom converged in          31  iterations
     for nz =  14
     delta  =    8.7051691277235455E-017

 Orbital energies
 n   kappa   l   occupancy            energy
 1 -1   0      2.0000000E+00 -1.3070214E+02
 2 -1   0      2.0000000E+00 -1.0194340E+01
 3 -1   0      2.0000000E+00 -7.9837677E-01
 2  1   1      2.0000000E+00 -7.0556273E+00
 3  1   1      2.0000000E+00 -3.0662229E-01
 2 -2   1      4.0000000E+00 -7.0081090E+00
 3 -2   1      0.0000000E+00 -3.0422316E-01

  Total energy
     Total                    :    -578.91322954391410     
############################################
\end{verbatim}

\fbox{Example 5: \textsc{atompaw} input for Si -- r2SCAN01, non-relativistic,
using generalized Kohn-Sham}
\begin{verbatim}
Si 14
WTAU XC_MGGA_X_R2SCAN01+XC_MGGA_C_R2SCAN01 splr00.1d0 splns600 loggrid 2001
3 3 0 0 0 0
3 1 2
0 0 0
c
c
v
c
v
2
1.7   1.5   1.7    1.7
y
14
n
y
14
n
y
2
y
12
n
MODRRKJ VANDERBILTORTHO  Besselshape
3 0   VPSMATCHNC
1.7
1.7
1.7
1.7
1.7
1.7
ABINITOUT
default
XMLOUT
default
PWSCFOUT
UPFDX  0.0125d0   UPFXMIN   -7.d0     UPFZMESH 14.d0
PWPAWOUT
END

###Si#####
 Completed calculations for Si
 Exchange-correlation type:
Exchange functional (LibXC):
  J. W. Furness, A. D. Kaplan, J. Ning, J. P. Perdew, and J. Sun, J. Phys. Chem. Lett. 11, 8208 (2020)
  J. W. Furness, A. D. Kaplan, J. Ning, J. P. Perdew, and J. Sun, J. Phys. Chem. Lett. 11, 9248 (2020)
Correlation functional (LibXC):
  J. W. Furness, A. D. Kaplan, J. Ning, J. P. Perdew, and J. Sun, J. Phys. Chem. Lett. 11, 8208 (2020)
  J. W. Furness, A. D. Kaplan, J. Ning, J. P. Perdew, and J. Sun, J. Phys. Chem. Lett. 11, 9248 (2020)
 Full generalized Kohn-Sham equations solved
  Radial integration grid is logarithmic 
r0 =   4.3309254E-04 h =   6.0632956E-03   n =      2001 rmax =   8.0000000E+01
Splinesolver method used for bound states
Splinesolver grid parameters r0 and ns:
   0.10000      600
 Non-relativistic calculation
   AEatom converged in          19  iterations
     for nz =  14.00
     delta  =    3.9767450008378184E-017
  All Electron Orbital energies:         
  n  l     occupancy       energy
 1  0      2.0000000E+00 -1.3175185E+02
 2  0      2.0000000E+00 -1.0501811E+01
 3  0      2.0000000E+00 -8.1811500E-01
 2  1      6.0000000E+00 -7.1862608E+00
 3  1      2.0000000E+00 -3.0245126E-01

  Total energy
     Total                    :    -578.62529162965529     
 Completed calculations for Si
 Exchange-correlation type:
Exchange functional (LibXC):
  J. W. Furness, A. D. Kaplan, J. Ning, J. P. Perdew, and J. Sun, J. Phys. Chem. Lett. 11, 8208 (2020)
  J. W. Furness, A. D. Kaplan, J. Ning, J. P. Perdew, and J. Sun, J. Phys. Chem. Lett. 11, 9248 (2020)
Correlation functional (LibXC):
  J. W. Furness, A. D. Kaplan, J. Ning, J. P. Perdew, and J. Sun, J. Phys. Chem. Lett. 11, 8208 (2020)
  J. W. Furness, A. D. Kaplan, J. Ning, J. P. Perdew, and J. Sun, J. Phys. Chem. Lett. 11, 9248 (2020)
 Full generalized Kohn-Sham equations solved
  Radial integration grid is logarithmic 
r0 =   4.3309254E-04 h =   6.0632956E-03   n =      2001 rmax =   8.0000000E+01
Splinesolver method used for bound states
Splinesolver grid parameters r0 and ns:
   0.10000      600
 Non-relativistic calculation
   SCatom converged in           8  iterations
     for nz =  14.00
     delta  =    9.9941022151552512E-017
   Valence Electron Orbital energies:         
  n  l     occupancy       energy
 3  0      2.0000000E+00 -8.1811472E-01
 3  1      2.0000000E+00 -3.0245141E-01

  Total energy
     Total                    :    -578.62527041729822     
     Valence                  :    -46.479751073168650     
  paw parameters: 
       lmax =            2
         rc =    1.7016270216702749     
        irc =         1366
   rc_shape =    1.5072490061787436     
    rc_vloc =    1.7016270216702749     
    rc_core =    1.7016270216702749     
 Sequence of dataset construction steps modified for mGGA
 Only projectors from Vanderbilt scheme available
 Vloc: VPS match (norm-conservation) with l= 3;e=   0.0000E+00
 Projector type: modified RKKJ projectors + Vanderbilt ortho.
 Bessel compensation charge shape zeroed at   1.5072E+00

Number of basis functions     6
 No.   n    l      Energy         Cp coeff         Occ
    1    3    0 -8.1811472E-01  3.4456771E+00  2.0000000E+00
    2  999    0  1.4000000E+01  1.0928984E-01  0.0000000E+00
    3    3    1 -3.0245141E-01  7.7138408E+00  2.0000000E+00
    4  999    1  1.4000000E+01  2.3344063E-01  0.0000000E+00
    5  999    2  2.0000000E+00 -4.9695966E-01  0.0000000E+00
    6  999    2  1.2000000E+01 -2.2394019E-01  0.0000000E+00
 Completed diagonalization of ovlp with info =        0
  
 Eigenvalues of overlap operator (in the basis of projectors):
    1        1.70014131E-01
    2        1.05637034E+00
    3        1.68670242E+00
    4        1.10551313E+01
    5        2.25209551E+01
    6        2.73898074E+01
  
  Summary of PAW energies
        Total valence energy       -46.479751063306111     
          Smooth energy             11.559261831969753     
          One center               -58.039012895275860     
          Smooth kinetic            2.4995956007877114     
          Vloc energy             -0.91305325877548571     
          Smooth exch-corr         -2.2202015931524510     
          One-center xc            -39.215909951163795     
################
\end{verbatim}


\begin{center}
	{\bf{Some details of input file options}}


\end{center}	


\begin{itemize}
	\item[{\bf{Line 1}}]   This line contains two inputs only, separated
		with  space(s):   [Atomic symbol (a2)]    [Atomic number (i3)]
	\item[{\bf{Line 2}}]   This line is very complicated, starting with
		the exchange-correlation functional and followed by formalism
		and grid information.  The input is not case sensitive.

		The exchange-correlation functional can be one of several
		in-house routines or can reference libxc subroutines.

		The following keywords reference in-house routines:
		\begin{itemize}
			\item \verb+LDA-PW+
			\item \verb+GGA-PBE+
			\item \verb+GGA-PBESOL+
			\item \verb+HF+
			\item \verb+MGGA-R2SCAN-001+ -- Original r2SCAN (generalized Kohn-Sham equations)
			\item \verb+MGGA-R2SCAN-01+ -- Modified r2SCAN (generalized Kohn-Sham equations)
		\end{itemize}
		
		The following examples illustrate the syntax  used to access the libxc
		 routines using
		the names assigned on the libxc webpage
		\url{https://www.tddft.org/programs/libxc/functionals/}, each optionally
		prepended with  ``\verb+XC_+''.
		\begin{itemize}
			\item PBESOL:  \verb^GGA_X_PBE_SOL+GGA_C_PBE_SOL^
			\item r2SCAN01 (partial): \verb^XC_MGGA_X_R2SCAN01+XC_MGGA_C_R2SCAN01^\\
			   Note that in this case, only $V_{xc}(r)$ will be used in
			   the Kohn-Sham formulation.
			\item r2SCAN01 (complete): 
			\verb^WTAU XC_MGGA_X_R2SCAN01+XC_MGGA_C_R2SCAN01^\\
			   Note that in this case, the keyword ``WTAU'' means that the kinetic
			   energy density will be used and the generalized Kohn-Sham equations
			    will be solved self-consistently.
		\end{itemize}
		
		
		The remainder of line 2 controls the formalism (if not Kohn-Sham or
		generalized Kohn-Sham), the discretization grid, splinesolver parameters,
		and additional features.  The order of these entries are not important.
		Some of these are as follows:
		\begin{itemize}
		\item SCALARRELATIVISTIC --   Solve Harmon and Koelling scalar relativistic
		  equations  (does not work with generalized Kohn-Sham (meta-GGA))
		\item DIRACRELATIVISTIC  -- Solve Dirac equations for upper and lower 
		  radial wavefunctions (does not work with generalized Kohn-Sham (meta-GGA)
		  and only implemented for graphatom)
		\item SPLINEINTERP SPLR0xxxx  SPLNSnnnn -- Use splinesolver algorithm
		  to solve self-consistent Kohn-Sham or generalized  Kohn-Sham equations.
		  This is the default  for meta-GGA calculations, but the SPLINEINTERP 
		  keyword enables the splinesolver algorithm for other exchange-correlation
		  functionals. SPLR0xxxx, where xxxx is a real number (with or without
		  a space) allows for the adjustment of the $r0$ parameter of the spline
		  grid for values other than the default of $0.1d0$.
		    SPLNSnnnn, where nnnn is an integer (with or without
		  a space) allows for the adjustment of the $ns$ parameter of the spline
		  grid for values other than the default of $400$.
                  Note that setting $ns>400$ typically increases
                  the computation time because in the current version
                  of the code, all of the eigenvalues/eigenvectors
                  of the $n_s \times n_s$ matrix $\boldsymbol{\Lambda}$ are
                  calculated in each self-consistent cycle to
                  find the all-electron radial functions of the 
                  bound states. 
		\item LOGGRID  nnnn   --  Set the main discretization grid for the
		  calculation.   For example,  LOGGRID 2001 seems to work well in most cases.
		\end{itemize}
		
	\item[{\bf{Line 3}}]	$n_0$    $n_1$    $n_2$     $n_3$   $n_4$     $n_5$ \\
	Here $n_l$ represents the maximum principal quantum for each occupied or
	partially occupied shell $l$ or 0 if the shell $l$ is not relevant. For example, \\
	\verb# 4 4 3 0 0 0 #\\
	represents the configuration $1s^22s^23s^24s^22p^63p^64p^63d^{10}$, or the
	ground state of Kr.
	\item[{\bf{Line 4}}]
	The following information concerns line 4 and possibly several lines following.
	\begin{itemize}
	\item For non-relativistic or scalar relativistic calculations,  the following
	lines list \\
	$n \;\;  l \;\; $   occ  \\
	for each partially occupied shell, ending with the line
	\verb^ 0  0  0 ^ \\
	For example, for Kr, line 4 reads\\
	 \verb^ 0  0  0 ^ \\
	 For Cu in the configuration  $1s^22s^23s^24s^12p^63p^64p^03d^{10}$,
	 lines 4, 5, and 6 read\\
	  \verb^ 4  0  1 ^ \\
	   \verb^ 4  1  0 ^ \\
	    \verb^ 0  0  0 ^ \\
	\item For Dirac equation calculations, the following lines list \\
	$n \;\;  l \;\; \kappa  \;\;$ occ \\
	for each partially occupied shell, ending with the line
	\verb^ 0  0  0  0^ \\
		For example, for Kr, line 4 reads\\
	 \verb^ 0  0  0  0 ^ \\
	 For Cu in the configuration  $1s^22s^23s^24s^12p^63p^64p^03d^{10}$,
	 lines 4, 5, 6, and 7 read\\
	  \verb^ 4  0 -1  1.d0 ^ \\
	   \verb^ 4  1  1  0.d0 ^ \\
	    \verb^ 4  1 -2  0.d0 ^ \\
	    \verb^ 0  0  0    0^ \\
	    Note that for each shell $l \kappa$,   the maximum occupancy is
	    $2|\kappa|$.   Also note that $\kappa$ is related to the total 
	    angular momentum quantum number $j$ according to
	     $\kappa = \pm \left(j+\frac{1}{2} \right)$.
	  \end{itemize}
	
	    
\end{itemize}
The remaining lines of the input file are very similar to older versions of
the code.



\end{document}
