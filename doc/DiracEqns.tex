\documentclass[11pt]{article}
\special{papersize=8.5in,11in} \pagestyle{plain}
\setlength{\oddsidemargin}{0in} \setlength{\evensidemargin}{0in}
\setlength{\textwidth}{6.5in} \setlength{\parindent}{0in}
\setlength{\parskip}{0.2in} \setlength{\topmargin}{-0.2in}
\setlength{\headsep}{0in} \setlength{\headheight}{0in}
\setlength{\textheight}{9.5in} \setlength{\jot}{0.2in}

\usepackage{url}
\usepackage{amsmath}

\input{../../../../shortcuts}
\begin{document}

\cc{{\LARGE{\bf{Notes for Dirac equations and 
scalar-relativistic equations used in the 
 {\em{atompaw}} code.}}}}
\cc{N. A. W. Holzwarth, Wake Forest University, Winston-Salem, NC 27106
  (natalie@wfu.edu)\\
 \today}

\cc{Scalar-relativistic treatment}
The scalar-relativistic equations were originally developed by Koelling and 
Harmon\cite{koelling:1977} as a way to represent the physics of the Dirac
equation, average over spin-orbit components. Another good reference for
these equations is on the NIST website\cite{levine:1997} \\
http://physics.nist.gov/PhysRefData/DFTdata/contents.html.   In terms of
Rydberg
energy units
(where the fine structure constant $\alpha$ is related to the speed of light
according to $c = 2/\alpha$).   Explicitly, the constants expressed
 in SI units are as follows: 
$\alpha=4 \pi \epsilon_0 e^2/(\hbar c)$, the unit of
length is the Bohr radius, $a_B= 4 \pi \epsilon_0 \hbar^2/(m_e e^2)$,
and the Rydberg energy is $\varepsilon_{Ry}= \alpha^2 m_e c^2/2$ ). 
In these units, the differential equation satisfied by 
upper component of the radial wavefunction $(G(r)/r$ with quantum number
$\kappa$ is
\ee{\left( \frac{d^2}{dr^2} - \frac{\ell(\ell+1)}{r^2} \right) G(r) + 
M(r)(E-V(r)) G(r) - \frac{M^{\prime}(r)}{M(r)} \left( \frac{d}{dr}
+ \frac{\langle \kappa \rangle}{r} \right) G(r) = 0.   \label{SC}}
Here,
\ee{M(r) \equiv 1 + \left( \frac{\alpha}{2} \right)^2 (E - V(r)).}
Here we are interested  in the spin-orbit splitting pair for each
$\ell > 0$, where $\kappa=-\ell-1=-(j+\half)$ corresponds to
$j=\ell+\half$ and $\kappa=\ell=(j+\half)$ corresponds to
$j=\ell-\half$.
The orbital angular momentum averaged value of  $\kappa$ is given by
\ee{ \langle \kappa \rangle = \frac{1}{2(2 \ell + 1)} \left( \ell ( 2 \ell )
+ ( -\ell -1) (2 \ell + 2 ) \right) = -1.}

Shadwick, Talman, and Normand\cite{talman:1989a} showed that it is possible to
transform this equation into a form without the first derivative so that the
Numerov integration scheme can be applied:
\ee{y(r) = \frac{G(r)}{\sqrt{M(r)}},}
with the resulting differential equation:
\ee{\frac{d^2}{dr^2} y(r) = A(r) y(r),}
\begin{eqnarray}
\displaystyle{A(r) \equiv \frac{\ell ( \ell + 1)}{r^2}
+ (V(r) - E) M(r) + \frac{3}{4} \left( \frac{\alpha}{2} \right)^4
\left( \frac{1}{M(r)} \frac{dV(r)}{dr} \right)^2} \\ \nonumber
\displaystyle{+ \frac{1}{2} 
\left( \frac{\alpha}{2} \right)^2 \frac{1}{M(r)} \frac{d^2 V(r)}{dr^2}
+ \left( \frac{\alpha}{2} \right)^2 \frac{1}{r M(r)} \frac{d V(r)}{dr}}.
\end{eqnarray}

We have programmed these equations, and find that they work reasonably well
with LDA exchange-correlation functionals, but for GGA functionals their
sensitivity to the potential tends to lead to uncontrolled oscillations.   
Consequently, with the help of Marc Torrent and Francois Jollet of CEA, who
modified a code of David Vanderbilt, we have developed the following alternate
approach.    In the Vanderbilt code, the second-order scalar-relativistic
code is written in terms of two coupled first order equations of the form:
\ee{\frac{d}{dr} G(r) = \frac{G(r)}{r} + M(r) F(r).\label{first}}
\ee{\frac{d}{dr} F(r) = - \frac{F(r)}{r} + \left( 
\frac{\ell (\ell + 1)}{r^2} \frac{1} {M(r)} - (E - V(r)) \right) 
G(r).\label{second}}

In these expressions, $F(r)$ is an auxiliary function which is similar to
the radial function of the lower component in the Dirac equation.
In this form, the scalar-relativistic equations are much more stable, since
they do not directly use derivatives of the potential.   Even with this
approach, we find that the scalar-relativistic equations are sensitive to
mesh size for the GGA exchange-correlation form.   
For a point nucleus, the electron-nucleus interaction has the
singular form $-2Z/r$.   
In order to use the Vanderbilt
code, it is necessary to to evaluate the wavefunction at several points
at the origin and in the asymptotic regions.    For $r \rightarrow 0$,
we can make a power series expansion  using
\begin{equation}
M(r) \underset{r \rightarrow 0}{\approx}
 1 +\left(\frac{\alpha}{2} \right)^2 \left( \frac{2Z}{r}+
E - V_0
 - V'_0 r \right) \label{smallr}
\end{equation}
The corresponding
 form of the radial wavefunctions near the origin takes
the form
\begin{equation} \label{upsmall}
G(r) \underset{r \rightarrow 0}{\approx} r^{\gamma} \left(C_0 + C_1 r  +C_2 r^2\right) 
\end{equation}
and, from Eq. (\ref{first}) the corresponding auxiliary function takes
the form
\ee{F(r) \underset{r \rightarrow 0}{\approx} \frac{r^{\gamma}}{r M(r)}\left( C_0 (\gamma -1) + 
C_1 \gamma r
+ C_2 (\gamma+1) r^2 \right).\label{dnsmall}}
In these expressions, 
the coeffients can be determined by analyzing Eq. (\ref{SC}) 
according to powers of $r$.
For example,
\ee{M(r)(E-V(r))  \underset{r \rightarrow 0}{\approx} 
\frac{1}{r^2} \alpha^2 Z^2+
\frac{1}{r} Z\left(2+\alpha^2 (E-V_0)\right)+
r^0 \left( (E-V_0) + \frac{\alpha^2}{4} \left(
(E-V_0)^2 -4ZV'_0 \right) \right)....}
\ee{-\frac{M'(r)}{M(r)}  \underset{r \rightarrow 0}{\approx}
\frac{1}{r} + r^0 \left(-\frac{2}{Z \alpha^2} \left( 1+ \frac{\alpha^2}{4}
(E-V_0) \right) \right) + r^1 \left( \frac{4}{\alpha^2 Z^2} +
\frac{V'_0}{Z} + \frac{E-V_0}{\alpha^2 Z^2} \left(2+\frac{\alpha^2}{4}
(E-V_0) \right) \right)+ .....}
The leading radial power coefficient is given by
\begin{equation}
\gamma= \sqrt{ \ell(\ell+1) + 1 -Z^2 \alpha^2 }.
\end{equation}
From the $r \rightarrow 0$ behavior of Eq. (\ref{SC}), the
relationship between the coefficients $C_n$
takes the general form:
\ee{T_n^{(-2)}C_{n} + T_n^{(-1)} C_{n-1} + T_n^{(0)} C_{n-2}=0,}
where
\ee{T_n^{(-2)}=n(2 \gamma +n),}
\ee{T_n^{(-1)} = Z(2+\alpha^2(E-V_0)) - \frac{2}{\alpha^2 Z} \left( 1 +
\frac{\alpha^2}{4} (E-V_0) \right)(\gamma+n-1),}
and
\ee{T_n^{(0)}=-\alpha^2 Z V'_0+\left(1+
\frac{\alpha^2}{4}(E-V_0)\right)(E-V_0) +
\left( \frac{V'_0}{Z} + \frac{4}{Z^2 \alpha^4} \left( 1 + 
 \frac{\alpha^2}{4} \left(E-V_0\right) \right)^2 
\right)(\gamma+n-1).}
With these parameters, we can determine
\ee{C_1= - \frac{T_0^{(-1)} C_0}{T_1^{(-2)} } \;\;\;{\rm{and}}\;\;\;\;
C_2= - \frac{T_1^{(-1)} C_1+T_0^{(0)} C_0}{T_2^{(-2)}}.\label{rec}}

The code also implements several finite
nuclear models described by Andrae\cite{andrae:2000} which has no
singular behavior at the origin.   
For the finite nuclear models, the singular electron-nuclear term is 
omitted, while the values of the potential constants $V_0$ and $V'_0$ 
are adjusted accordingly.  In this case, we can use the same small $r$
expansions as in Eqs. \ref{upsmall} and \ref{dnsmall}, using
the recursion formulas Eq. (\ref{rec}), however the
parameters are altered to 
\ee{\gamma \rightarrow \tilde{\gamma}= \ell + 1.}
\ee{T_n^{(-2)} \rightarrow \tilde{T}_n^{(-2)}=n(2 \ell +1+n).}
\ee{T_n^{(-1)} \rightarrow \tilde{T}_n^{(-1)}= \frac{\alpha^2}{4} V'_0
\left(\frac{1}{\left(1 + \frac{\alpha^2}{4}(E-V_0) \right)}\right)(\ell+n).}
\ee{T_n^{(0)} \rightarrow \tilde{T}_n^{(0)}= (E-V_0)
\left(1 + \frac{\alpha^2}{4}(E-V_0) \right) +
 \left( \frac{\alpha^2}{4} V'_0
\left(\frac{1}
{\left(1 + \frac{\alpha^2}{4}(E-V_0) \right)}\right)\right)^2 (\ell+n)
.}
Note that these equations are slightly inconsistent with the
scalarrelativistic code versions $< 4.0.1.0$ for the $C_2$ coefficients.
In order to find bound states $(0 > E \equiv -b^2$, the solver requires
inward integration  from $r_{\rm max}$.   Here, we assume that the
potential vanishes up to a possible (positive) Coulombic charge of $qe$
so that
\ee{V(r)\underset{r \rightarrow \infty} {\approx} = -\frac{2 q}{r}.}
Then the asymptotic form of the radial wavefunctions are
\ee{G(r) \underset{r \rightarrow \infty} {\approx}
 \e{-br}r^g,\;\;\;\;\;{\rm{and}}\;\;\;\;\;\;
F(r) \underset{r \rightarrow \infty} {\approx}
\left(-b + \frac{g-1}{r} \right) \frac{G(r)}{M(r)}, }
where
\ee{b \equiv \sqrt{|E|\left(1-\frac{\alpha^2|E|}{4}\right)},}
and
\ee{g \equiv \frac{q}{b} \left(1- \frac{\alpha^2}{2}|E| \right).}






\cc{Full Dirac treatment}
It is also possible to analyze the full Dirac equation in a similar
way.    For this, we follow the convention of Loucks\cite{Loucks:1965}
and define the function 
\begin{equation}
cF(r) \equiv \frac{2}{\alpha} F(r).
\end{equation}

The function $cF(r)$ is used for input to and within the subroutine
unboundD and boundD, but the output lower component wave function is
$F(r)$.
The coupled equations for the full Dirac radial functions in
Rydberg energy units then become

\begin{equation} \label{cFG}
\left( \frac{d}{dr} + \frac{\kappa}{r} \right) G(r) =
       \left( 1 +\left( \frac{\alpha}{2} \right)^2 \left(E - V(r))\right)
           \right) cF(r)
\end{equation}
and
\begin{equation}
\left( \frac{d}{dr} - \frac{\kappa}{r} \right) cF(r) =
  -\left(E - V(r))\right) G(r)
\end{equation} 
In order to determine the form of the radial wavefunctions in the
limit as $r \rightarrow 0$, we again expand the potential as a power
series in $r$:
\begin{equation}
V(r) \underset{r \rightarrow 0}{\approx} -\frac{2Z}{r}+V_0, \label{zero}
\end{equation}
and also represent the upper and lower radial wavefunctions in
a power series with leading power coefficient $r^s$:
\begin{equation}
G(r)  \underset{r \rightarrow 0}{\approx} r^s \sum_{n=0}^{\infty} A_n r^n \;\;\;\;\;
cF(r)  \underset{r \rightarrow 0}{\approx} r^s \sum_{n=0}^{\infty} B_n r^n.
\end{equation}
A recursion formula  for the coefficients takes the form
\begin{equation}
\left( \begin{array}{cc} s+n+\kappa & -Z \alpha^2/2 \\
                        2Z & s+n-\kappa  \end{array} \right)
\left( \begin{array}{c} A_n \\
                       B_n \end{array} \right)
  = \left( \begin{array}{cc} 
                0 &      1+\alpha^2(E-V_0)/4 \\
                  -(E-V_0)  & 0 \\  \end{array}  \right)
   \left( \begin{array}{c} A_{n-1} \\ B_{n-1} \\ \end{array} \right).
\label{rec1}
\end{equation}

The condition for a non-trivial solution fixes the value of the
leading power $s$:
\begin{equation}
s= \sqrt{\kappa^2-Z^2 \alpha^2},
\end{equation}
and the corresponding ratio of the leading coefficients is given
by 
\ee{B_0 = \frac{2(s+\kappa)}{Z\alpha^2} A_0.}
From evaluating Eq. (\ref{rec}), the first order coefficients can
be computed to be
\ee{A_1=\frac{ 4 \alpha^4 Z^2 + \left( 4 \alpha^2 +E -V_0 \right)
\left( (s+\kappa) -2 \alpha^2 Z^2 \right)}{2Z(2s+1)} A_0,}
and
\ee{B_1=-\frac{  \left( 2 \alpha^2 +E -V_0 \right)
\left(2 (s+\kappa) +(E-V_0) \right)}{(2s+1)} A_0.}
With these results, we can use the following relationships to initialize
the radial wavefunctions near the origin:
\ee{G(r) \underset{r \rightarrow 0}{\approx} r^s(A_0 +A_1 r) \;\;\;\;
{\rm and}\;\;\;\; 
cF(r) \underset{r \rightarrow 0}{\approx} r^s(B_0 +B_1 r).}
Note that for the case of a finite potential model or a pseudopotential
where  $Z=0$ in Eq. (\ref{zero}), this analysis has to be re-examined since
the expansion coefficients diverge.

For evaluating the aymptotic form, 
we can assume that the potential vanishes so that the upper component
satisfies the equation
\ee{\left( \frac{d^2}{dr^2} - \frac{\ell(\ell+1)}{r^2}  - b^2 \right)
G(r)=0, }
where 
\ee{b= \sqrt{|E|\left(1 - \frac{\alpha^2}{4}|E| \right)}}
and $\ell=\kappa$ for $\kappa>0$ and 
and $\ell=-\kappa-1$ for $\kappa<0$.  
We can thus assume the asymptotic solution form:
\ee{G_{n \kappa}(r) \underset{r \rightarrow \infty} {\approx}
k_{\ell}(br),}
where $k_{\ell}(u)$ denotes $u$ times a modified bessel function
of the third kind, having the values
\ee{k_0(u)= \e{-u} \;\;\; k_1(u)= \e{-u} \left( 1 + \frac{1}{u} \right)
\;\;\; k_2(u)= \e{-u} \left( 1 + \frac{3}{u} + \frac{3}{u^2} \right),}
etc.
These functions can be generated according to the recurrsion 
relation\cite{HBMF}
\ee{k_{\ell+2}(u)=k_{\ell}(u)+\frac{(2 \ell +3)}{u} k_{\ell+1}(u).}
From the recurrsion relations\cite{HBMF}, one can also show that
\ee{\left( \frac{d}{du} + \frac{\ell}{u} \right) k_{\ell}(u)
= - k_{\ell-1}(u),}
which can be used to obtain the asymptotic form of the lower component
\ee{cF_{n \kappa}(r)\underset{r \rightarrow \infty}{\approx} -\frac{|E|}{b}
k_{\ell-1}(br).}
Note that the text, {\em{Atomic Structure Theory: Lectures on 
Atomic Physics}} by Johnson\cite{johnson_atomic_2007} gives many more 
details on the asymptotic forms and a later version of this code
should include those results. At the moment, the code
only includes the leading terms quoted here and will not work for an
ion.
    Johnson\cite{johnson_atomic_2007}
also works out a very clever method of converging the energy 
eigenvalues from the inward and outward integration.   In particular,
he noted that the solutions of Eq. (\ref{cFG}) with labels 1 and 2 
corresponding to energies $E_1$ and $E_2$ have the identity:
\ee{\frac{d}{dr} ( G_1\; cF_2 - G_2\; cF_1) = (E_1-E_2) 
\left( \frac{\alpha^2}{4} cF_1\;cF_2 +G_1 G_2 \right).\label{slope}}
In a similar way that Hartree\cite{Hartree} used the mismatch
of the slope of the inward and outward integration results to adjust
the energy, Eq. (\ref{slope}) can be used to correct the energy.   At
the matching radius $r_m$, we normalize the upper component so that
$G_{\rm out}(r_m)=G_{\rm in}(r_m)$, but the corresponding lower component
will have a discontinuity $\Delta [cF(r_m)] \equiv 
cF_{\rm out}(r_m)-cF_{\rm in}(r_m)$.
The first order estimate of the energy correction is thus given by
\ee{ \Delta E  =  C\; G(r_m) \Delta [cF(r_m)],
  \;\;\;\;{\rm where}\;\;\;\; C^{-1} \equiv \int_0^{\infty} \left(
(G(r))^2 + \frac {\alpha^2}{4} (cF(r))^2 \right).}


In order to test whether the code is working, it is helpful to check
the analytic solutions for a H-like ion with nuclear charge $Z$,
which is given in the textbook by Bethe and Saltpeter\cite{BandS}
and can be expressed in terms of confluent hypergeometric polynomial 
(Kummer) functions\cite{HBMF} $F(a,b,z)$, the principal quantum
number 
\ee{n = |\kappa|, |\kappa| +1, |\kappa| + 2, .... \equiv n'+|\kappa|,}
the apparent principal quantum number
\ee{N \equiv \sqrt{n^2 -2n'(|\kappa|-s)},}
the normalization factor
\ee{{\cal{N}} \equiv \sqrt{\frac{\Gamma(2s+n'+1)}{(n'!) 4 N (N-\kappa)}}
\frac{1}{\Gamma(2s+1)}\left(\frac{2Z}{N}\right)^{1/2},}
and the length parameter (in Bohr units)
\ee{ \rho \equiv \frac{2Zr}{N}.}
In terms of the previously defined parameters, including the
energy $E$ relative to the electron rest mass energy (in
Rydberg energy units), the
normalized radial wavefunctions take the form  (up to an arbitrary sign
convention)
\ee{G_{n \kappa}(r)= {\cal{N}} \sqrt{\left(2+
\frac{\alpha^2}{2}E_{n \kappa}\right)}
\;\e{-\rho/2} \rho^s 
\left((N-\kappa) F(-n',2s+1,\rho) 
-n' F(-n'+1,2s+1,\rho) \right)-}
and
\ee{F_{n \kappa}(r)=- {\cal{N}} 
\sqrt{\left(-\frac{\alpha^2}{2}E_{n \kappa}\right)}
\;\e{-\rho/2} \rho^s 
\left((N-\kappa) F(-n',2s+1,\rho) 
+n' F(-n'+1,2s+1,\rho) \right).}
The energy eigenvalue $E_{n \kappa} < 0$ is given by:
\ee{E_{n \kappa} = \frac{2}{\alpha^2} \left( \left(1+ \frac{\alpha^2 Z^2}
{(n-|\kappa|+s)^2} \right)^{-1/2} -1 \right).}

In order to take advantage of the solver properties, it is convenient
to calculate all of the bound states for each $\kappa$ value.   These
are called in the order of $\kappa = -1, 1, -2, 2, -3, 3, -4 ..$ 
corresponding to orbital angular momenta of the upper component
$\ell = 0, 1, 1, 2, 2, 3, 3, ...$.    If another ordering is
desired, this can easily be accomplished with the use of a mapping
algorithm.   

\cc{Example input and output}
At the moment, only graphatom is programmed to use the Dirac equation
solver.     To run graphatom, use the following command in a directory
containing the ``in'' file:

\begin{verbatim}
[path]/bin/graphatom<in>&out&
\end{verbatim}

An example ``in'' file for Bi is:

\begin{verbatim}
Bi  83             
LDA-PW  diracrelativistic loggrid    2001
6 6 5 4 0 0
6 1 1 2
6 1 -2 1
0 0 0 0
0
\end{verbatim}

The corresponding Bi.GA file is:

\begin{verbatim}
 
 Completed calculations for Bi
 Perdew-Wang LDA -- PRB 45, 13244 (1992)
  Radial integration grid is logarithmic 
r0 =   8.3003918E-05 h =   6.8893252E-03   n =      2001 rmax =   8.0000000E+01
 Dirac-relativistic calculation
   AEatom converged in          29  iterations
     for nz =  83
     delta  =   3.313784838079538E-016
 
 Orbital energies
 n   kappa   l   occupancy            energy
 1 -1   0      2.0000000E+00 -6.6480533E+03
 2 -1   0      2.0000000E+00 -1.1934355E+03
 3 -1   0      2.0000000E+00 -2.8796157E+02
 4 -1   0      2.0000000E+00 -6.5980525E+01
 5 -1   0      2.0000000E+00 -1.1514073E+01
 6 -1   0      2.0000000E+00 -1.0379038E+00
 2  1   1      2.0000000E+00 -1.1468243E+03
 3  1   1      2.0000000E+00 -2.6688214E+02
 4  1   1      2.0000000E+00 -5.6821818E+01
 5  1   1      2.0000000E+00 -8.3277130E+00
 6  1   1      2.0000000E+00 -4.2545518E-01
 2 -2   1      4.0000000E+00 -9.7647279E+02
 3 -2   1      4.0000000E+00 -2.2874691E+02
 4 -2   1      4.0000000E+00 -4.7404690E+01
 5 -2   1      4.0000000E+00 -6.4975188E+00
 6 -2   1      1.0000000E+00 -2.8622974E-01
 3  2   2      4.0000000E+00 -1.9414140E+02
 4  2   2      4.0000000E+00 -3.2493774E+01
 5  2   2      4.0000000E+00 -2.0654190E+00
 3 -3   2      6.0000000E+00 -1.8605221E+02
 4 -3   2      6.0000000E+00 -3.0736387E+01
 5 -3   2      6.0000000E+00 -1.8441098E+00
 4  3   3      6.0000000E+00 -1.1362662E+01
 4 -4   3      8.0000000E+00 -1.0958823E+01
 
  Total energy
     Total                    :    -46803.6120503352     
\end{verbatim}





\newcommand{\bibdir}{../../../../bibfiles/}
% Create the reference section using BibTeX
\bibliographystyle{unsrt}
\bibliography{\bibdir methods,\bibdir exx,specific}








\end{document}
